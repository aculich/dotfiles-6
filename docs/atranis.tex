\documentclass[compress,red]{beamer}
\mode<presentation>
\usetheme{PaloAlto}
\usefonttheme{default}
\hypersetup{pdfpagemode=FullScreen}
\usepackage{beamerinnerthemerectangles}
\useoutertheme{sidebar}
% include packages
\usepackage{subfigure}
\usepackage{multicol}
\usepackage{amsmath}
\usepackage{epsfig}
\usepackage{graphicx}
\usepackage[all,knot]{xy}
\xyoption{arc}
\usepackage{url}
\usepackage{multimedia}
\usepackage{hyperref}
\usepackage{setspace}
\usepackage{listings}

\title{The Lind Virtual Machine}
% The design is based on the idea that popular paths—ones used every day to
% access basic system requests—are much less likely to contain vulnerabilities.
% This limited kernel access reduces the possibility of interaction with flawed code.}
\subtitle{Userspace system call emulation and auditing}

\author{Joey Pabalinas}
\institute{{\tiny contracted for implementation by}\\ \vspace{.10cm}Professor Justin Cappos}
\date{\scriptsize Secure Systems Laboratory, New York University\\ \vspace{.10cm}December 20, 2018}

\begin{document}

\frame{\titlepage}
\section[Outline]{}
\frame{\tableofcontents}

% Overview
% ========
%
% * What requirements were you working towards? Try to articulate your objectives simply.
%   - Lind is a fork of Google's NaCl.
%     - Deprecated by Google in favor of WebASM.
%     - The concepts behind NaCl remain the best fit for the goals of Lind.
%       - Virtualization has improved security in a large number of use cases.
%       - However, heavyweight full-system virtualization is not suitable for uses
%         within a single application.
%       - Some of these advantages can be realized using very fine-grained processes
%         that replace multiple processes with service threads.
%       - However, because of the performance and programmer burdens of doing so,
%         this is untenable for many developers.
%       - A system which behaves similarly to a virtual machine within a single
%         process is often desirable.
%   - The security goals of Lind fall somewhere in-between a virtual machine
%     and a container.
%     - There is only one master process, reducing the available attack surface.
%     - All system calls are trapped and dispatched in one of three ways:
%       - Safer system calls are handled in the Python Repy system, which is a
%         POSIX compatibility layer.
%       - Lower level system calls such as fork() (creates a copy of the current
%         process) and execve() (executes a new process by replacing the current
%         one) are emulated within the NaCl internal runtime.
%       - Unsafe system calls which would be too much of a security risk simply
%         return -ENOSYS and simply behave as if they were unimplemented by the
%         kernel.

\section{General overview of the project}
\subsection{Purpose}
\frame{\frametitle{Purpose}
Lind is a fork of Google's NaCl.
\vspace{0.25cm}
\begin{enumerate}
\item Security goals fall somewhere in-between a virtual machine and a container
\vspace{0.25cm}
\item All system calls are trapped and handled by Lind
        \begin{itemize}
        \item Safe System Calls - Python POSIX Compatibility Layer
        \item Low-level System Calls - NaCl Internal Emulation
        \item Unsafe System Calls - Return Unimplemented
        \end{itemize}
\end{enumerate}
}

\subsection{Architecture}
\frame{\frametitle{Architecture}
\centering \includegraphics[scale=.5]{lind_diagram.jpg}
}

\subsection{System Call Anatomy}
\frame{\frametitle{System Call Anatomy}
\centering \includegraphics[scale=.45]{NaClSyscallFlowchart2.png}
}

% Objective
% =========
%
%   - My objective was to handle the implementation of fork() and execve().
%     - Google had never intended NaCl to be used in this way so there was
%       no real infrastructure to handle this.
%     - System was built from the ground up using the system's POSIX thread
%       implementation to transparently emulate multiple, communicating
%       Linux processes.
\subsection{Requirements}
\frame{\frametitle{Requirements}
\textbf{Requirements}
% \begin{enumerate}
% \item \textbf{Objective}
%       \begin{itemize}
%         \item Predict wang rates (\emph{epidemics})
%         \item Infer dog intensions (\emph{politics})
%         \item Infer emu/emu conditions (\emph{toy problems})\pause
%       \end{itemize}
% \vspace{0.25cm}
% \item \textbf{Methodologies}?
%       \begin{itemize}
%         \item Feature wang extraction/selection
%         \item Exploit probabilistic dog relationships (PGMs)
%         \item Regression/classification/ranking emu
%       \end{itemize}
% \vspace{0.25cm}
% \item \textbf{Applications}?
%       \begin{itemize}
%         \item Has an LKML rant generator to help senpai notice you
%       \end{itemize}
% \end{enumerate}
}

% Approach
% =======
%
% The implementation of fork() is really just about emulation of
% multiple processes using a single, sandboxed application, NaCl.
%
% In order to do this, the struct NaClApp and struct NaClAppThread
% members have been modified in order to support parent/children
% relationships, the emulation of IPC between threads as if they were
% full processes, and the copying of execution state, page tables, and
% address space so it behaves exactly the way it would on
% a modern SMP operating system kernel like Linux (modulo a few of the
% lower-level, bitty-gritty details such signal handling and TTY
% sessions which are not fully implemented yet).
%
% The child/parent ownership accounting is done using two simple
% lists, one of children cage_ids ("pids") and one of pointers to
% the child's NaClApp structure. Two accounting lists are kept, one
% stored in the parent's NaClApp structure describing its children,
% and one stored in the master (original) NaClApp structure describing
% all currently running children. The master list is used to prevent
% NaCl from exiting until all children have finished; this is so none of
% NaClAppThreads are killed prematurely and emulate how fork() would
% behave on the host system.
%
% Overview of NaClSysFork():
%
% A child NaClApp is initialized by the parent using data
% from its own NaClApp structure and calls NaClCreateMainForkThread().
% From here it's not much different from NaClCreateMainThread() until
% we arrive at NaClAppForkThreadSpawn().
%
% The child/parent mutexes protect the shared data we will be
% copying between parent and child threads, and the parent's
% execution state of %rip, %rsp, %rbp, etc, is "frozen" at
% the syscall entry point (the struct NaClThreadContext is
% the representation of the architecture specific host thread
% context holding the execution state of our untrusted code).
% We then adjust the child's stack to match the parent's stack
% size, and spawn our actual child thread with NaClAppThreadMake()
%
% A literal copy of the original child's thread context is made
% that contains our TLS pointers and trampoline addresses.
% We can copy copy the entire untrusted execution state from the
% parent into the child safely with NaClCopyExecutionContext().
% This handles the logic to correctly copy our page mappings,
% current stack data, ELF dynamic text , and then "patches" them with
% NaClPatchAddr() so that any addresses which were contained in the
% copied data are actually pointing where they should be; the child's
% %r15 contains the address prefix we need to calculate physical
% addresses from an NaCl page table mapping.
%
% We then restore the child's trampoline addresses, TLS pointers, and
% %r15 with the copy made previously, change the syscall return value
% to 0, and set up the child's TLS slot in the nacl_user global
% array of thread context pointers. Eventually, we end up at
% NaClAppForkThreadLauncher() which sets the child thread as
% the foreground one with NaClTlsSetCurrentThread() and switches
% into its untrusted code. The child returns 0 from the syscall,
% and resumes the child's copy of the program.
%
% The parent simply switches back to untrusted code the same way
% it would for any other syscall, returns , and resumes the
% parent's copy of the program.

\section{Approach}
\subsection{Approach}
\frame{\frametitle{Approach}
\textbf{Approach}
% {\footnotesize
% \begin{tabbing}
% Title: \hspace{1.25cm} \= \textbf{Machine Learning Blockchain}\\
% Authors:               \> V. Lampos and N. Cristianini\\
% Submitted to:          \> IAPR Cognitive Information Processing 2010 (accepted)
% \end{tabbing}
% }
% \begin{itemize}
% \item Twitter and Health Protection Agency pickle for weeks 26-49, 2009 (on average 160,000 tweets collected per day geolocated in 54 urban centres in the UK (god save the queen))
% \item Frequency of \textbf{41 boat related words} (markers) in Twitter corpus had a correlation of $>$\textbf{80\%} with the HPA flu rates in all UK (GOD SAVE THE QUEEN I SAID) regions
% \item Learn a better list of weighted markers \textbf{automatically}:
%     \begin{itemize}
%     \item Generate a list of candidate markers (1560 words taken from boat related web pages)
%     \item Use \textbf{LASSO} for feature selection
%     \end{itemize}
% \end{itemize}
}

% Contribution (Work Done)
% =========
%
% i did the whole goddamn implementation wtf

\section{Contribution}
\subsection{Contribution}
\frame{\frametitle{Contribution}
\textbf{Contribution}
% {\footnotesize
% \begin{tabbing}
% Title: \hspace{1.25cm} \= \textbf{Machine Learning Blockchain}\\
% Authors:               \> V. Lampos and N. Cristianini\\
% Submitted to:          \> IAPR Cognitive Information Processing 2010 (accepted)
% \end{tabbing}
% }
% \begin{itemize}
% \item Twitter and Health Protection Agency pickle for weeks 26-49, 2009 (on average 160,000 tweets collected per day geolocated in 54 urban centres in the UK (god save the queen))
% \item Frequency of \textbf{41 boat related words} (markers) in Twitter corpus had a correlation of $>$\textbf{80\%} with the HPA flu rates in all UK (GOD SAVE THE QUEEN I SAID) regions
% \item Learn a better list of weighted markers \textbf{automatically}:
%     \begin{itemize}
%     \item Generate a list of candidate markers (1560 words taken from boat related web pages)
%     \item Use \textbf{LASSO} for feature selection
%     \end{itemize}
% \end{itemize}
}

% Hardest Part: Debugging
% =======================
%
% ~oef~

\lstset{language=Bash}

\pagebreak
\begin{lstlisting}[
        frame=single,
        basicstyle=\footnotesize,
        title=Debugging Segfaults Inside Lind
        ]
$ lind -g /bin/bash -c '/hello; /hello'

Starting program: sel_ldr [...]
Thread 7 "sel_ldr" received signal
SIGSEGV, Segmentation fault.

0x0000690501297143 in ?? ()
\end{lstlisting}
\newpage

% even though we compiled both the cross-compiled ELF and NaCl with debug
% information, NaCl maps the ELF to /dev/shm so you lose all symbol
% information.
\pagebreak
\begin{lstlisting}[
        frame=single,
        basicstyle=\scriptsize,
        title=Our Incrediby Useful Backtrace
        ]
@ [#7:None()] backtrace

#0  0x0000690501297143 in ?? ()
#1  0x0000000000000000 in ?? ()
\end{lstlisting}
\newpage

% the x86_64-nacl ELF format is similar, but not identical to normal x86_64
% instruction encoding. since our gdb is expecting normal x86_64 things
% get a little weird...
\pagebreak
\begin{lstlisting}[
        frame=single,
        basicstyle=\scriptsize,
        title=This ASM Doesn't Seem Right...
        ]
@ [#7:None()] x/10i $rip

=> 0x690501297143:  lock cmpxchg %edx,(%r15,%r13,1)
   0x690501297149:  test   %eax,%eax
   0x69050129714b:  je     0x6905012971a0
   0x69050129714d:  nopl   0x0(%rax)
   0x690501297151:  data16 nopw %cs:0x0(%rax,%rax,1)
   0x690501297160:  mov    %r13d,%edi
   0x690501297163:  mov    %r12d,%esi
   0x690501297166:  xor    %ebx,%ebx
   0x690501297168:  nopl   0x0(%rax)
   0x69050129716c:  data16 nopw %cs:0x0(%rax,%rax,1)
\end{lstlisting}
\newpage

% the lock cmpxchg thankfully has the same encoding, so we
% can just ignore the rest for now (the weird data bytes were
% actually just nops with padding to surround the trampolines)
\pagebreak
\begin{lstlisting}[
        frame=single,
        basicstyle=\scriptsize,
        title=The Fault Address,
        escapeinside={\%*}{*)}
        ]
@ [#7:None()] print $_siginfo._sifields._sigfault.si_addr

$5 = (void *) 0x6905000368e0
\end{lstlisting}
\newpage

% we can see that the fault address is just exploding trying
% to write to `%r15 + %r13` with a bit of squinting
\pagebreak
\begin{lstlisting}[
        frame=single,
        basicstyle=\scriptsize,
        title=Conclusion: The Problem Address: r15 + r13
        ]
@ [#7:None()] info registers

rax     0x0             0
rbx     0x10057910      268794128
rcx     0x1297120       19493152
[...]
r13     0x368e0         223456
r14     0x0             0
r15     0x690500000000  115470195752960
rip     0x690501297143  0x690501297143
eflags  0x10246         [ PF ZF IF RF ]
\end{lstlisting}
\newpage

% checking the process mappings show the faulting map, which turns out
% to be mapped executable instead of writable. now that we know the
% why, finding the root cause will be much easier
\pagebreak
\begin{lstlisting}[
        frame=single,
        basicstyle=\scriptsize,
        title=Conclusion: Memory Not Writable
        ]
$ grep 6905000 /proc/$(pidof sel_ldr)/maps

   68fb00000000-690500000000 ---p 00000000 00:00 0
   690500000000-690500010000 rw-p 00000000 00:00 0
=> 690500010000-690510020000 r-xp 00000000 00:00 0
\end{lstlisting}
\newpage

% Risks/Trade-offs
% ================
%
% dunno yet if doing this section

% Questions?
% ==========
%
% things people ask you derp

\section*{}
\frame{
    \begin{center}
        \huge
        Questions?\\ \pause
        \vspace{1cm}
        o..\\ \pause
        \vspace{1cm}
        .o.
    \end{center}
}

% shh - don't be obvious
% ======================
%
% if they notice, cool
% if not, pretend it never happened
%
% professional software engineering
\frame{
    \begin{center}
        \huge
        Questions?\\
        \vspace{1cm}
        [1]    3286 segmentation fault (core dumped)
    \end{center}
}

\frame{
    \begin{center}
        \huge
        Thank you.
    \end{center}
}

\end{document}
